\title{Representational Considerations in Models of Language Change and Stability}
\author{Rebecca L. Morley}
\subtitle{}
\renewcommand{\lsSeries}{tgdi}
\renewcommand{\lsSeriesNumber}{}

\BackBody{Research in linguistics, as in most other scientific domains, is usually approached in a modular way – narrowing the domain of inquiry considerably, in order to allow for increased depth of study. This is both necessary, as well as productive, for a topic as wide-ranging and complex as human language. However, precisely because language is a complex system, tied to perception, learning, memory, and social organization, the assumption of modularity is also an obstacle to understanding linguistic competence at a deeper level. Modularity, in fact, can both create artificial theoretical problems, as well as obscure real theoretical difficulties. This book examines the consequences of enforcing non-modularity along two dimensions: the temporal, and the cognitive. Along the temporal dimension, synchronic and diachronic domains are linked by the requirement that sound changes must lead to viable, stable language states. Along the cognitive dimension, sound change and variation are linked to speech perception and production by requiring non-trivial transformations between acoustic and articulatory representations.

Methodologically, the focus of this work is on computational modeling. Computational implementation is, by far, the most reliable method for detecting under-determined elements of a given theory. Implementation is impossible without complete explicitness, and thus exposes gaps and covert assumptions in our theories. Specifically, in terms of linked modules, implementation confronts us with the necessity of transforming the theoretical primitives of one sub-domain (e.g., synchronic representations), into those of another sub-domain (e.g., diachronic representations). What implementation alone cannot do, however, is impose theoretical coherence. I argue that is necessary to formally assess the functional equivalence of specific implementational choices, as well as their mapping to theoretical structures.

The present work applies this analytic approach to a series of implemented models of sound change, identifying the source of theoretical inconsistencies and incompatibilities between modules. As problems are discovered, possible solutions are proposed, incrementally constructing a set of sufficient properties for a working model. Because internal theoretical consistency is enforced, this model corresponds to an explanatorily adequate theory. And because explicit links between modules are required, this is a theory, not only of sound change, but of many aspects of phonological competence.

This work highlights two aspects of modeling work that receive relatively little attention: the formal mapping from model to theory, and the scalability of demonstration models. Focusing on these aspects of modeling makes it clear that any theory of sound change in the specific is impossible without a more general theory of language: of the relationship between perception and production, the relationship between phonetics and phonology, the learning of linguistic units, and the nature of underlying representations. In fact, theories of sound change that fail to address these aspects of language are, in actuality, making tacit, untested, assumptions about their properties. Developing specific hypotheses about these myriad aspects of speech may appear to bring overwhelming complexity and difficulty to the linguist's task. However, placing one component within the larger system of which it is a part can actually facilitate our study of that one component. In effect, these connected systems impose boundary conditions of ecological validity that reduce the theoretical search space.

This work argues for the necessity of a deep commitment to anti-modularity; to rigorous tests of the explicitness of theory (implementability); and to rigorous tests of implemented models. These actions, in tandem, are argued to lead to more global theoretical insights – insights that are impossible when certain aspects of the complex system of language are treated as black boxes. It is my hope that these results will encourage greater integration of computational modeling and linguistic theory, rather than the former occupying a somewhat separate and, for some, inaccessible, niche. Even very simple models – such as many in this book – have the power to demonstrate where our intuitions are wrong, and where even the most unquestioned assumptions of linguistic theory are, in fact, paradoxical in nature.}
%\dedication{Change dedication in localmetadata.tex}
\typesetter{Felix Kopecky, Rebecca L. Morley}
% \proofreader{}

\BookDOI{}%ask coordinator for DOI
\renewcommand{\lsISBNdigital}{}
\renewcommand{\lsISBNhardcover}{}
\renewcommand{\lsSeries}{cfls} % use lowercase acronym, e.g. cfls, sidl, eotms, tgdi
\renewcommand{\lsSeriesNumber}{99} %will be assigned when the book enters the proofreading stage
\renewcommand{\lsID}{251} % contact the coordinator for the right number
