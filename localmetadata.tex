\title{Representational considerations in models of language change and stability}
\author{Rebecca L. Morley}
\subtitle{}

\BackBody{Research in linguistics, as in most other scientific domains, is usually approached in a modular way – narrowing the domain of inquiry in order to allow for increased depth of study. This is necessary and productive for a topic as wide-ranging and complex as human language. However, precisely because language is a complex system, tied to perception, learning, memory, and social organization, the assumption of modularity can also be an obstacle to understanding language at a deeper level. 

The methodological focus of this work is on computational modeling, highlighting two aspects of modeling work that receive relatively little attention: the formal mapping from model to theory, and the scalability of demonstration models. A series of implemented models of sound change are analyzed in this way. As theoretical inconsistencies are discovered, possible solutions are proposed, incrementally constructing a set of sufficient properties for a working model. Because internal theoretical consistency is enforced, this model corresponds to an explanatorily adequate theory. And because explicit links between modules are required, this is a theory, not only of sound change, but of many aspects of phonological competence.}
%\dedication{Change dedication in localmetadata.tex}
\typesetter{Felix Kopecky, Rebecca L. Morley}
% \proofreader{}

\BookDOI{10.5281/zenodo.3264909}%ask coordinator for DOI
\renewcommand{\lsISBNdigital}{978-3-96110-190-0}
\renewcommand{\lsISBNhardcover}{978-3-96110-191-7}
\renewcommand{\lsSeries}{cfls} % use lowercase acronym, e.g. cfls, sidl, eotms, tgdi
\renewcommand{\lsSeriesNumber}{4} %will be assigned when the book enters the proofreading stage
\renewcommand{\lsID}{251} % contact the coordinator for the right number
